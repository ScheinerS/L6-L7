\chapter*{Agradecimientos}

Desde que me desperté con la iniciativa de escribir esto, me vienen a la cabeza múltiples recordatorios de personas a quienes debería agradecer por haberme acompañado en este proceso, ya sea desde lo profesional como desde lo personal. Son muchísimas y muy variadas así que espero no olvidarme de nadie!

Por empezar, quería agradecer especialmente a Ana Dogliotti, mi directora. Ana es una persona brillante, meticulosa y apasionadamente dedicada a la profesión. Esto no sólo significa que trabaja arduamente para seguir perfeccionando los resultados de su trabajo, sino que también siempre está colaborando activamente con la gente que la rodea. A lo largo de mi estadía doctoral puedo asegurar que nunca me faltó nada de lo que busqué o necesité, y esto se lo debo muy especialmente a ella, sobre todo teniendo en cuenta que, incluso en una circunstancia económica muy desfavorable del país, mi doctorado estuvo excepcionalmente nutrido de montones de experiencias laborales en el exterior, congresos, estadías y tareas de campo que sin la ayuda de Ana habrían sido completamente inviables. Más allá de esto, siempre estuvo ahí para ayudarme a resolver todos los conflictos y las dudas que fui teniendo - de hecho, no recuerdo una sola vez que no haya podido dejar un segundo lo que estaba haciendo para ayudarme, o un solo mail que no me haya respondido! Su ejemplo a lo largo de estos años fue fundamental para ir creciendo en todo sentido y por eso y por su solidaridad le estoy muy agradecido.

Otros mentores a quienes les estoy muy agradecido y a quienes tengo mucho respeto son mis codirectores: Kevin Ruddick (putativo) y Fran Grings (formal). De ellos aprendí a encarar los conflictos que fui teniendo en el doctorado con tranquilidad y sobre todo con optimismo. Antes, cuando las cosas no daban como esperaba me frustraba rápidamente. Con Ana y con ellos aprendí un enfoque mucho más optimista y científico: el de aprender con tezón - y fundamentalmente, con tranquilidad - de los errores. En resumen: entender que las situciones en que las cosas no dan como uno espera son las mejores oportunidades para seguir aprendiendo. También a ellos les estoy muy agradecido y les debo la concreción de este trabajo.

Luego, a lxs demás compas del grupo de Teledetección aparte de Ana y Fran, a quienes siempre pude acudir ante cualquier duda o necesidad que surgió, y con quienes seguiré compartiendo almuerzos por un tiempo. En mayor o menor medida, todxs ellxs me enseñaron cosas y fueron dándome una mano siempre que la necesité: Pablo Perna, Mer Salvia, Esteban Roitberg, Vero Barraza, Mati Barber, Marian Franco, Estefi Piegari, Vane Douna, Juli Villa, Wally Rava y a lxs que conocí y ya no están en el IAFE: Haydée Karszenbaum, Cin Bruscantini, Fede Carballo, entre otrxs.

También agradezco a todas las personas con quienes trabajé circunstancialmente durante esta tesis: a lxs colegas del REMSEM de RBINS en Bruselas; a Caro Sá y Giulia Sent - con quienes trabajé y me divertí mucho en mi estadía en Lisboa; a Claudia Simionato y Diego Moreira (y otrxs) del DCAO; a Ivi Tropper y Caro Tauro de CONAE; a Laura Sánchez de Limnología - con quien compartí muchas campañas en Chascomús; a Robert Frouin de Scripps - quien ideó las bases del esquema por PCA; a Bruno Lafrance de LOA - quien nos proporcionó comentarios útiles sobre el código CNES-SOS; a Natalia Morandeira y nuevamente Ivi y lxs compas de tele - quienes me ayudaron en las mediciones de detección de hidrocarburos; a Ana Delgado de Bahía Blanca, y a varias personas más.

También a lxs docentes y compañerxs de la escuela del verano (boreal) de 2016 en Villefranche-sur-Mer, que me hizo aprender y crecer muchísimo y conocer mucha gente brillante de muchas partes del mundo con quienes espero trabajar en algún futuro próximo. Y también a todxs lxs colegas con quienes no trabajo directamente pero con quienes compartí charlas científicas y no científicas en diferentes estadías fuera del IAFE.

También a mis amigxs del IAFE: las charlas aerográficas con Mecha, los mates dulces de Maizu, los mates con burrito de Chomi, las pepas marca Día de Choni, entre otrxs. Y a mis amigxs de la facu, compañía fundamental en todo esto: lxs de Física: Tefi - mi cellista preferida, Fredi - el de los estornudos raros, la Colo, Esteban, Euge, Alan-Facu, Telita, Fiore, Marcos, Virgi, Vale. Y lxs compas atmosféricxs y asociadxs: Luz, Mili, Lula, Manu, Iae, Ine y lxs de Dinámica de los Océanos. Y muchxs más.

A mis amigxs del Sur. A la charla que arranqué con Augusto en 2005 hablando de la Geografía de la Tierra Media y que ahora sigue en formato epistolar (aunque ya no hablamos de la Tierra Media). A la pasión por la música que comparto con Belu y a su caos determinista. Al humor y la fortaleza de Coco, y a sus gustos específicos y a sus personajes. A las tardes de tarot y río con Guido. A los deliciosos muffins de Chela y las coreografías de Britney de Martina, y las charlas reflexivas con ambas. También a mis amigos del colegio y a Maite y lxs poetas de los sábados.

A mi abuela Isabel, y a mi abuelita del corazón, Celia. A mi tía Silvita - la rebelde educadora de la familia - y a mis primas Julia y Lucía.

A mamá Alicia y a papá Horacio. A ellxs debo gran parte lo que soy, con sus virtudes y defectos. De ellxs aprendí lo que es el amor incondicional y a ellxs les agradezco la inmensa labor que implicó criarnos y vernos crecer a mis hermanxs y a mí. Y a ellxs, mis hermanxs, a quienes también en otro sentido les debo la vida y lo que soy: Ale, Pepo, Ine y Clari. Y a mis cuñadas y a Fede, Isa y la sobri que se viene. Por muchas tardes más de juego y aventura con ellxs.

A la memoria de Doña Irma Adela Marangón de Laguzzi, la abuela más tierna que se haya podido tener. A sus cantos friulanos y argentinos, a sus historias de guerra y a su sonrisa. Bastará por ahora con cerrar los ojos y recordarla cantando y silbando los tangos de Gardel y Libertad Lamarque, Il General Cadorna y Quel Mazzolin di Fiori.

A la Piru, quien un día, hace ya 7 años, cayó hacia mí desde un ombú y me abraza cada mañana y cada noche. Y con quien espero seguir compartiendo más y más mates con menta, y nuevas recetas de cocina, y nuevos chismes de último momento, y nuevas alegrías y tristezas, por donde sea que vayamos en este futuro abierto e inmenso que tenemos por delante.

A lxs que mencioné y me olvidé, siempre con fe, amor y esperanza, lxs abraza,

$\quad$

Juancho