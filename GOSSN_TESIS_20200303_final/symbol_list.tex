\chapter{Listado de Símbolos}
\label{sym}

{\parindent0pt

$A(\lambda)$: Factor de calibración de algoritmos de turbidez (\S \ref{dat:s:dog15}) y SPM (\S \ref{dat:s:spmNech10})

$A_{X}$: Coeficiente de la banda $X$ sobre BLR

$a$: Coeficiente de absorción

$a^{*}$: Segunda componente del espacio de color $La^{*}b^{*}$

$a_{p}^{*}$: Coeficiente específico de absorción de partículas

$a_{g}^{*}$: Coeficiente específico de absorción de CDOM o \textit{gelbstoff}

$a_{CHL}^{*}$: Coeficiente específico de absorción de la clorofila-a

$a_{PC}^{*}$: Coeficiente específico de absorción de la ficocianina

$a_{w}$: Coeficiente de absorción del agua líquida pura

$a_{i}$: Proyección $i$-ésima en la descomposición de la señal por PCA

$\alpha_{\lambda_{i},\lambda_{j}}$: Cociente de reflectancias del agua en $\lambda_{i}$ y $\lambda_{j}$

$\alpha_{\tau}$: Parámetro de Angstrom de $\tau_{a}$

$b$: Coeficiente de dispersión

$b_{b}$: Coeficiente de retrodispersión

$b^{*}$: Tercera componente del espacio de color $La^{*}b^{*}$

$b_{b,p}^{*}$: Coeficiente específico de retrodispersión de partículas

$b_{b,w}$: Coeficiente de retrodispersión del agua líquida pura

$BL(\rho_{X})(\lambda_{M} |\lambda_{L},\lambda_{R})$: Altura de Línea de Base en de la reflectancia de tipo $X$ trazada entre $\lambda_{L}$ y $\lambda_{R}$ en $\lambda_{M}$

$BLR(\rho_{X})(\lambda_{L}, \lambda_{M}, \lambda_{R})$: BLR de la reflectancia de tipo $X$ en el triplete de bandas en longitudes de onda $\lambda_{L}$, $\lambda_{M}$ y $\lambda_{R}$

$\beta$: Función de Dispersión en Volumen

$C_{0}$: Pepita o \textit{nugget} de un variograma

$c$: Coeficiente de extinción

$c_{\tau}$: Escala de decaimiento exponencial de $\tau_{a}$

$c_{\rho}$: Escala de decaimiento exponencial de $\rho_{a}$

$[C]$: Concentración de la componente $C$

$C(\lambda)$:  Factor de calibración de algoritmos de turbidez (\S \ref{dat:s:dog15}) y SPM (\S \ref{dat:s:spmNech10})

$Cond(A)$: Número condicional de la matriz A

$d$: Tamaño característico de una partícula

$\frac{dV(r)}{d(lnr)}$: Distribución volumétrica de tamaños de partículas

$dV_{a}/dlnr$: Distribución volumétrica de tamaños de aerosoles

$\delta_{r}$: Cociente de depolarización molecular

$E$: Irradiancia

$E_{u}$: Irradiancia ascendente

$E_{d}$: Irradiancia descendente

$e^{PCA}_{j}$: $j$-ésimo autovector de la matriz de varianza-covarianza

$\epsilon_{\lambda_{i},\lambda_{j}}$: Cociente de reflectancias de aerosoles en $\lambda_{i}$ y $\lambda_{j}$

$F_{0}$: Irradiancia solar extraatmosférica

$f$: Fracción de la superficie cubierta por \textit{whitecaps} (WC)

$f'/Q$: Factor de corrección de superficie de Morel y Gentili 1996, \cite{morel1996}

$\Phi$: Flujo radiante

$\phi$: Ángulo acimutal relativo

$\varphi$: Ángulo acimutal de dispersión

$g_{glitter}$: Distribución de inclinaciones de facetas de la superficie

$\gamma$: Coeficiente global del modelo de reflectancia qSSA ($\gamma = \pi \Re f'/Q$)

$\gamma_{i}(h)$: Variograma de imagen en banda $i$

$H_{a}$: Altura característica de aerosoles

$H_{r}$: Altura característica molecular

$h$: Desfasaje entre dos píxeles

$I_{pol}$: Índice de polarización (CNES-SOS)

$L$: Radiancia

$L$: Primera componente del espacio de color $La^{*}b^{*}$

$L_{TOA}$: Radiancia a TOA

$L_{r}$: Radiancia debida a dispersión Rayleigh

$\lambda$: Longitud de onda

$\lambda_{X}$: Longitud de onda en determinado rango o banda $X$

$m_{0}$: Régimen lineal en el modelo de Michaelis-Menten

$m_{a}$: Índice de refracción de aerosoles (parte imaginaria)

$\mu$: Factor geométrico de masa de aire

$N$: Número de bandas correctoras en los esquemas SWIR-PCA (\S \ref{pca})

$n_{a}$: Índice de refracción de aerosoles (parte real)

$n_{w}$: Índice de refracción relativo aire-agua

$n_{max}$: Máximo orden de dispersión (CNES-SOS)

$\mathcal{N}(0,\sigma)$: Variable aleatoria normal de media 0 y desvío estándar $\sigma$

$\Omega$: Ángulo sólido

$\omega$: Factor de transición rojo-NIR en el algoritmo de Dogliotti et al. 2015, \cite{dogliotti2015}

$P$: Presión atmosférica

$P_{VF}$: Fracción de área de un píxel ocupada por vegetación flotante

$R$: Distancia Sol-Tierra (en UA)

$R_{0}$: Distancia Sol-Tierra media (= 1 UA)

$R_{rs}$: Reflectancia sensada remótamente

$R_{rs,X}$: Reflectancia sensada remótamente del \textit{endmember} $X$

$\Re$: Factor de corrección de superficie de Loisel y Morel 2001, \cite{loisel2001}

$\rho_{w}$: Reflectancia del agua

$\rho_{w}^{X}$: Reflectancia del agua estimada a partir del sensor $X$ o la corrección atmosférica $X$

$\rho_{r}$: Reflectancia debida a la dispersión Rayleigh

$\rho_{a}$: Reflectancia debida a aerosoles

$\rho_{ra}$: Reflectancia debida a dispersión múltiple Rayleigh-aerosoles

$\rho_{g}$: Reflectancia debida al \textit{sunglint}

$\rho_{wc}$: Reflectancia debida a la espuma (\textit{whitecaps})

$\rho^{*}_{wc}$: Reflectancia debida a la espuma (\textit{whitecaps}) a $f=1$

$\rho_{RC}$: Reflectancia corregida por dispersión Rayleigh

$\rho_{TOA}$: Reflectancia a TOA

$\rho^{SOS}$: Reflectancia simulada con el código CNES-SOS

$\rho_{s}$: Reflectancia corregida por Rayleigh en el \textit{software} SeaDAS

$S_{\infty}$: Régimen de saturación en el modelo de Michaelis-Menten

$\sigma_{i}$: Amplitud del ruido en la banda $i$-ésima

$T$: Transmitancia directa

$T\_{X}$: Turbidez medida/estimada con el sensor/método $X$

$t$: Transmitancia difusa

$t_{g}$: Transmitancia por absorción de gases atmosféricos

$t_{BLR}$: Transmitancia equivalente de BLR

$\tau$: Espesor óptico

$\tau_{r}$: Espesor óptico de dispersión Rayleigh de la atmósfera

$\tau_{a}$: Espesor óptico de aerosoles

$\Theta$: Ángulo entre las direcciones de dispersión e incidente

$\theta_{s}$: Ángulo cenital solar

$\theta_{v}$: Ángulo cenital de observación

$w$: Intensidad del viento en superficie

$\hat{x}$: Valor estimado de la variable $x$

$z$: Altura respecto de la interfase agua-aire

$z = 0^{+}$: Magnitud medida justo sobre la interfase agua-aire

$z = TOA$: Magnitud medida a tope de la atmósfera

}
