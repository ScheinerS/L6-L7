\chapter{Conclusiones y perspectivas}
\label{con}

Esta tesis tuvo como objetivo general aportar al conocimiento preexistente en el campo del sensoramiento remoto de color del mar en aguas ópticamente complejas, con particular énfasis en el diseño e implementación de métodos novedosos de corrección atmosférica en el área del Estuario del Río de la Plata.
%
Así como en otras disciplinas basadas en la teledetección del entorno terrestre, el diseño, la calibración y la validación de algoritmos de estimación de las propiedades biogeofísicas a partir de la radiación electromagnética que arriba al sensor se respalda fuertemente en la disponibilidad de mediciones de campo de dichas variables. Es por esto que una parte fundamental de esta tesis consistió en la adquisición de datos \textit{in situ} en el área de interés - el Río de la Plata - en conjunto con la puesta a punto de instrumentos bioópticos de campo y el procesamiento, el análisis de calidad y la sistematización de las mediciones en una base de datos (capítulo \ref{dat}).

En particular, tomando el Río de la Plata como área de estudio, esta tesis se enfocó en la estimación de la reflectancia del agua a partir de la radiación que arriba a tope de la atmósfera, es decir, en el diseño de algoritmos de corrección atmosférica. Históricamente, el problema de la corrección atmosférica en el área de color del mar fue abordado mediante la utilización de bandas espectrales en el infrarrojo cercano (NIR) donde, en el caso de aguas ópticamente claras como las del Mar de Sargaso, era válido suponer que la señal medida provenía enteramente de la atmósfera de forma tal que la existencia de dichas bandas posibilitaba separar la señal atmosférica y la del agua. Luego, una vez aislada la señal atmosférica en el NIR, y mediante la existencia de una base de simulaciones de tranferencia radiativa que abarcaran las bandas del NIR y las de interés (visible, VIS), la implementación de un esquema de extrapolación del NIR al VIS era posible.
%
Sin embargo, a medida que las concentraciones de material ópticamente activo en el agua (como fitoplancton, partículas minerales o detrito orgánico) aumentan, dicha hipótesis comienza a perder validez, implicando la necesidad de implementar nuevas estrategias basadas en nuevos supuestos. Tal como se detalla en el capítulo \ref{int}, la necesidad de obtener estimaciones de reflectancias del agua de calidad en regiones de aguas ópticamente complejas dio lugar a la creación de múltiples esquemas novedosos de corrección atmosférica que, o bien se respaldan en supuestos físicos de creciente complejidad sobre bandas específicas de corrección - de las que luego se extrapolará la señal atmosférica -, o bien en la creación de esquemas de espectro completo, muchos de los cuales se basan en métodos de inteligencia artificial - cuyos parámetros carecen de una interpretación física - optimizados a partir de conjuntos de entrenamiento. 

Estas nuevas familias de algoritmos muchas veces no proveen buenas estimaciones en el área del Río de la Plata - en particular en el frente de turbidez ubicado en la región de la Barra del Indio - ya que, o bien se vulneran incluso los nuevos supuestos físicos - por ej. el supuesto de agua negra en algunas bandas del infrarrojo de onda corta (SWIR) - o bien sus popiedades ópticas caen por fuera de los conjuntos de calibración/entrenamiento utilizados en los esquemas basados en inteligencia artificial.
%
Dada esta disyuntiva, el enfoque de esta tesis se orientó hacia la generación de nuevos esquemas de corrección atmosférica del estilo de la primera familia de algoritmos, es decir, basándonos en nuevas hipótesis físicas simples. Se optó de esta manera dado que, si bien el potencial operativo de los enfoques de inteligencia artificial o espectro completo es plenamente reconocido, los enfoques extrapolativos cuentan con la ventaja de que cada paso del proceso tiene una base física clara, lo que facilita la identificación y resolución de eventuales inconvenientes. Por ejemplo, un modelo de reflectancia de agua deficiente, una calibración imprecisa o un modelo de aerosoles inapropiado se harán evidentes con un enfoque extrapolativo, pero pueden ser menos obvios con un enfoque de sistema agua-atmósfera acoplado de espectro completo.

La puesta en órbita del sensor OLCI-A (Ocean and Land Colour Instrument) a bordo de Sentinel-3A en abril de 2016 dio inicio a una nueva era en el sensoramiento remoto de color del mar. Si bien la mayoría de sus bandas espectrales provienen del precursor MERIS (MEdium Resolution Imaging Spectrometer, 2002-2012), OLCI posee una banda novedosa en el SWIR, centrada en 1016 nm, que brinda una nueva configuración espectral, ideal para el desarrollo de nuevos algoritmos de tipo extrapolativo respaldados en nuevas hipótesis físicas. En el capítulo \ref{blr} se propuso la hipótesis de la cuasi-linealidad de la señal atmosférica en rangos espectrales cortos en bandas OLCI en el Rojo-NIR-SWIR - incluyendo la banda de 1016 nm - a partir de la cual se desarrolló, implementó y validó un esquema de corrección atmosférica para aguas turbias (aquí denominado BLR-AC: BaseLine Residual - Atmospheric Correction). En este capítulo se demostró que el rendimiento del algoritmo BLR-AC se compara favorablemente con otros algoritmos de corrección atmosférica existentes (el procesador estándar ESA/OLCI, el procesador SeaDAS implementado con las bandas OLCI 865 nm y 1016 nm y el esquema de espectro completo basado en redes neuronales C2RCC), particularmente en aguas extremadamente turbias, donde dichos algoritmos fallaron o mostraron una correlación no física entre imágenes de reflectancia de aerosoles y del agua. Aunque el esquema BLR-AC fue diseñado y testeado para imágenes OLCI, el mismo podría expandirse fácilmente a otros sensores que posean tripletes de bandas espectralmente cercanas en el Rojo/NIR/SWIR, como la misión conjunta Argentina-Brasileña SABIA-Mar, el sensor MSI a bordo de Sentinel-2, o sensores hiperespectrales como HICO o PROBA/CHRIS.

Durante el desarrollo de este esquema de corrección atmosférica, se hizo evidente, a partir de la visualización de las primeras imágenes OLCI disponibles para la región del Río de la Plata, la inesperada aparición de ruido de tipo \textit{sal y pimienta} en las imágenes, que se halló finalmente asociado a Eventos de Partículas Veloces (EPVs) causados por la influencia de la Anomalía Magnética del Atlántico Sur en la región - en la cual yace el Río de la Plata. El hecho de que este es el único cuerpo de agua turbia de gran extensión que se halla inmerso dentro del área de influencia de la Anomalía del Atlántico Sur, nos condujo, tanto por necesidad como por exploración, a la tarea imprescendible de implementar un algoritmo de detección y remoción de EPVs sobre imágenes OLCI en nuestra área de estudio. El desarrollo y testeo de dicho algoritmo es descrito en el capítulo \ref{ppe}, mostrando un buen desempeño y una buena correspondencia con predicciones preexistentes que caracterizaron - previo al lanzamiento - al ruido producido por EPVs.

Otra misión que inspiró la implementación de nuevas hipótesis de trabajo es SABIA-Mar (Satélite Argentino-Brasileño para InformAción del Mar, lanzamiento programado para 2022), cuyo campo prioritario de observación engloba a las costas argentinas y brasileñas del Atlántico Sudoccidental, incluyendo al Río de la Plata. Esta misión posee bandas en el SWIR lejano al igual que sus precursores MODIS (MODerate resolutIon Spectral imager), y a su vez, una banda en 1044 nm, similar a la de OLCI en 1016 nm; por lo que sus características espectrales posibilitan la implementación de nuevos esquemas de corrección atmosférica de aguas turbias basados en nuevas hipótesis de trabajo. En particular, el esquema desarrollado y testeado teóricamente del capítulo \ref{pca} se basa en una metodología novedosa para relacionar la señal de aerosoles en el SWIR con la señal en las bandas NIR, basada en la descomposición en componentes principales de la señal atmosférica de un conjunto de simulaciones de transferencia radiativa en las bandas correctoras (SWIR) y a corregir (NIR, y eventualmente VIS). La diferencia esencial con los esquemas preexistentes es que en el esquema por PCA las magnitudes utilizadas para la extrapolación se obtienen a partir de la matriz de varianza-covarianza de un ensamble de simulaciones de transferencia radiativa de la atmósfera y no de parámetros preestablecidos como la amplitud de la señal y el cociente de las bandas correctoras. El esquema por PCA desarrollado en esta tesis muestra resultados plausibles en la estimación de la reflectancia del agua en el NIR, aunque no fue testeado en imágenes satelitales, sino en un conjunto de simulaciones de transferencia radiativa del sistema acoplado atmósfera-agua al que se introdujeron reflectancias del agua medidas \textit{in situ} en diversas regiones del Río de la Plata.

Aparte de ser un cuerpo singularmente extenso de aguas ópticamente complejas, el Río de la Plata es un sitio donde frecuentemente se presentan eventos extraordinarios cuya fenomenología es muchas veces observable a partir de sensores de color del mar. Por ejemplo, entre enero y mayo de 2016 - es decir dentro del transcurso del período de esta tesis - ocurrió una significativa invasión temporaria de vegetación flotante en la región norte del estuario que causó una interrupción significativa de las actividades humanas desarrolladas en su entorno debido a la obstrucción de tomas de agua potable, bloqueo de puertos y marinas e introducción de animales peligrosos de humedales lejanos en el litoral de la Ciudad de Buenos Aires y alrededores. Este episodio condujo a la necesidad de desarrollar un algoritmo alternativo que permitiera la detección y cuantificación de vegetación flotante en la región y que a su vez pudiera discernir entre píxeles contaminados con vegetación flotante y regiones de aguas turbias - algo que los índices preexistentes como el FAI (Floating Algal Index) o el NDVI (Normalized Difference Vegetation Index) no logran. El desarrollo e implementación de dicho algoritmo (denominado FAIT, es decir, FAI para aguas Turbias) es descrito en el capítulo \ref{cam}, donde a su vez queda demostrada la plausibilidad de su implementación y una fuerte correlación entre el área de cobertura de la vegetación flotante y la anomalía positiva del caudal del estuario.

Por otro lado, otro evento que ha ocurrido en el pasado en el Río de la Plata - en enero de 1999 frente al Partido de Magdalena - y cuya probabilidad de reincidencia en la región es alta - debido al elevado tráfico naval - es el derrame de hidrocarburos. Entre el momento del derrame de Magdalena y la actualidad, la tecnología satelital de los sensores de color del mar ha progresado considerablemente, por lo que es fundamental un avance concomitante en el conocimiento de la impronta óptica que generan los derrames hidrocarburíferos, particularmente en un entorno de aguas turbias, de forma tal de poder brindar a la sociedad herramientas de caracterización y cuantificación frente a la ocurrencia de nuevos siniestros. En el capítulo \ref{oil} se describió un procedimiento experimental en el que se consiguió reproducir una muestra de aguas del Río de la Plata emulsionada con diferentes espesores de hidrocarburos a escala de laboratorio y bajo condiciones de iluminación natural. Sobre esta muestra se realizaron mediciones radiométricas a partir de las cuales se establecieron potenciales índices espectrales que permitieron discriminar la presencia de hidrocarburos respecto de la variabilidad natural asociada a diferentes concentraciones de sedimentos en aguas turbias del Río de la Plata.

El trabajo realizado en esta tesis presenta varias interrogantes que aún deben ser tratadas a futuro. Si bien se lograron generar algoritmos de corrección atmosférica bajo nuevas hipótesis de trabajo en las bandas del Rojo/NIR/SWIR, es fundamental testear qué impacto tendrá el error en la estimación de la señal atmosférica en el NIR sobre diferentes modelos de extrapolación en el VIS. Una posibilidad en la que actualmente se está trabajando para disminuir el error de extrapolación es la implementación de supuestos similares al de anclaje en el violeta/azul (\S \ref{int:s:ACUV}), bandas donde el error de extrapolación se magnifica, dado que este se incrementa a medida que la distancia espectral entre las bandas correctoras y a corregir es mayor.
%A su vez, es fundamental testear el esquema de corrección por PCA del capítulo \ref{pca} en sensores existentes como MODIS de forma tal de corroborar si la correspondencia entre los autovectores de PCA simulados y reales es óptima; caso contrario será necesario recalibrar los autovectores simulados a partir de datos satelitales.
Por otro lado, naturalmente, la adquisición de mediciones de campo junto con los métodos de procesamiento y sistematización de dicha información no son un problema cerrado: es fundamental dar continuidad a la base de datos bioópticos del Río de la Plata - ya sea a partir de nuevas campañas de mediciones como mediante la incorporación de nuevos instrumentos y nuevos esquemas de procesamiento de la información - y así seguir perfeccionando el conocimiento que se tiene actualmente en el área.

Si bien los resultados generados en el marco de esta tesis no son acabados y son perfectibles, el trabajo aquí presentado es un aporte importante al avance en el estudio del color del mar en aguas costeras y ópticamente complejas, como es el caso particular del Río de la Plata, en el contexto de un país que se encuentra actualmente desarrollando una misión satelital del color del mar y que tiene un creciente interés en el entendimiento y la explotación sustentable de sus recursos hídricos.