%\begin{center}
%\large \bf \runtitulo
%\end{center}
%\vspace{1cm}
\chapter*{\runtitulo}

\noindent
\textbf{Resumen}

El sensoramiento remoto en la región óptica del espectro electromagnético, o \textit{color del mar}, ha demostrado su capacidad de proveer información sinóptica de las propiedades ópticas y biogeoquímicas de los océanos. Esta se basa en la determinación de la radiancia espectral que proviene de la superficie del agua que es obtenida a partir de la señal que llega al tope de la atmósfera (TOA). La amplitud y la forma del espectro de este producto geofísico primario (generalmente dado como \textit{reflectancia}) son interpretadas en términos de los productos derivados como concentraciones de sustancias ópticamente activas o propiedades ópticas inherentes (\textit{Inherent Optical Properties}, IOPs) que pueden ser luego utilizadas en aplicaciones ambientales y modelos biogeoquímicos a escala regional y global. La precisión en la estimación de estos parámetros depende sin embargo de la capacidad de obtener la reflectancia medida justo sobre la superficie del agua, $\rho_{w}$, a partir de la radiancia total medida por el sensor, $L_{TOA}$. Este procesamiento que se le debe realizar a la señal incluye entre otros, la eliminación de la contribución de la atmósfera, proceso llamado \textit{corrección atmosférica} (CA). En este contexto, esta tesis tuvo como objetivo la exploración de nuevas alternativas de algoritmos de CA sobre la región abarcada por el Estuario del Río de la Plata (RdP), cuyas aguas extremadamente turbias (principalmente en la región del frente de turbidez, a la altura de la Barra del Indio) constituyen un desafío tanto para los esquemas de CA tradicionales, como para las alternativas desarrolladas para aguas turbias.


En el capítulo \ref{blr} se describe el algoritmo \textit{BLR-AC} (CA por Resíduos de Línea de Base), el cual fue calibrado, testeado y validado sobre imágenes OLCI (\textit{Ocean and Land Color Instrument}) aunque es potencialmente extendible a otros sensores con tripletes de bandas espectralmente cercanas en el rango Rojo-NIR-SWIR (NIR: \textit{Near Infrared} o infrarrojo cercano. SWIR: \textit{Short Wave Infrared} o infrarrojo de onda corta). Los resultados muestran desempeños favorables en comparación con otros esquemas (como BAC/BPAC v2.23, C2RCC, C2RCCnewNN, SeaDAS). Se observan correlaciones espaciales comparativamente más bajas entre las reflectancias del agua y las de aerosoles; valores más coincidentes con las mediciones realizadas \textit{in situ} (ejercicio de \textit{match-up}), y correspondencias muy buenas entre las reflectancias del agua en el Rojo/NIR estimadas con BLR-AC y con un esquema no operativo sencillo - diseñado \textit{ad-hoc} para la comparación - basado en la extrapolación de la señal de aerosoles en ventanas fijas de aguas claras a toda la región de interés. Si bien los resultados obtenidos con el esquema BLR-AC son favorables, aún es necesario extender la corrección a bandas del visible a partir de un método de extrapolación y eventuales condiciones de contorno adicionales en la región visible del espectro.

En el capítulo \ref{pca} se describe el algoritmo \textit{SWIR-PCA} (Análisis de Componentes Principales de la señal de aerosoles utilizando bandas en el SWIR) desarrollado para aguas ópticamente complejas del Río de la Plata para sensores como SABIA-Mar (Satélite Argentino-Brasileño para Información del Mar) y MODIS (MODerate Resolution Imaging Spectroradiometer) que poseen bandas en la región espectral del SWIR lejano, es decir, donde es válido el \textit{supuesto de agua negra} (Gordon 1978, \cite{gordon1978}) para cualquier concentración de sedimentos. El esquema está basado en la descomposición en componentes principales de la señal atmosférica en el NIR/SWIR y en la consecuente utilización de las bandas en el SWIR para la determinación de la señal en el NIR. El esquema fue testeado teóricamente a partir de simulaciones de transferencia radiativa de un sistema acoplado agua-atmósfera, mostrando resultados plausibles en la determinación de la reflectancias del agua en las bandas NIR de los sensores considerados (MODIS, SABIA-Mar, entre otros). A su vez, un análisis geoestadístico aplicado a un conjunto de imágenes Aqua/MODIS se realizó para testear el impacto del ruido del sensor sobre el desempeño del algoritmo, mostrando una baja sensibilidad del esquema al ruido del sensor. Si bien los resultados de dicho esquema son plausibles, aún es fundamental testear su desempeño en imágenes de sensores ya en órbita, como Aqua/MODIS y nuevamente extender la corrección a bandas del visible.

En el capítulo \ref{ppe} se describe un procedimiento de detección y remoción de eventos de partículas veloces (EPVs) que impactan en los sensores que orbitan en trayectorias de tipo LEO (\textit{Low Earth Orbit}) - como por ej., OLCI. Dichos eventos son extremadamente frecuentes en la región de la Anomalía Magnética del Atlántico Sur (SAA, \textit{South Atlantic Anomaly}) - en particular, afectando las imágenes del Río de la Plata. El desempeño de este algoritmo fue evaluado tanto visualmente como mediante la comparación con resultados preexistentes - evaluados sobre el sensor OLCI previo al lanzamiento. Los resultados indican que las imágenes sobre la región de la SAA contienen 27.8 veces más píxeles contaminados por EPVs que las correspondientes a regiones por fuera de la SAA y que las bandas más afectadas son las de 400 y 1016 nm, donde la fracción de píxeles detectados como contaminados por EPVs sobre la SAA alcanzó el 0.14\% y 0.26\%, respectivamente. Dicha corrección forma parte del preprocesamiento realizado previo a la corrección atmosférica descrita en el capítulo \ref{blr} (BLR-AC).

En el capítulo \ref{cam} se describe una aplicación desarrollada para imágenes de color del mar del Río de la Plata: el algoritmo FAIT (\textit{Floating Algal Index for Turbid waters}) de detección de vegetación flotante (VF) en aguas turbias. En este capítulo se evaluó la invasión inusual del jacinto de agua (\textit{Eichhornia crassipes}) ocurrida en el RdP en enero-abril de 2016 desde una perspectiva de múltiples misiones; se discutió el impacto de diferentes resoluciones espaciales en la detección de la VF, y finalmente se analizó y cuantificó su extensión espacial y variabilidad temporal y su relación con el caudal de salida del río.

Finalmente, en el capítulo \ref{oil} se describe otra aplicación específica a estas imágenes: un posible algoritmo de detección de derrames de hidrocarburos en aguas turbias, junto con diversos índices que permitirían determinar el espesor de la capa superficial del derrame. Para ello se armó una muestra de aguas del RdP a escala de laboratorio a la cual se vertieron diferentes volúmenes de hidrocarburos, y habiendo expuesto dicha muestra a condiciones de iluminación natural, se midieron las reflectancias del agua a diferentes espesores de derrame. Los resultados nos muestran que existen varios índices espectrales que podrían auspiciar de indicadores de presencia de hidrocarburos en aguas turbias, como el cociente de reflectancias roja e infrarroja, la luminosidad, la turbidez o el cociente violeta/rojo. Si bien las diferencias halladas entre las firmas de aguas con hidrocarburos y aguas del RdP a diferentes espesores de hidrocarburos son contundentes, es fundamental ampliar este estudio para contemplar el impacto de otras variables pertinentes, como la concentración de material particulado en suspensión, diferentes tipos de hidrocarburos y diferentes grados de emulsión.


\bigskip

\noindent\textbf{Palabras claves:} color del mar, corrección atmosférica, aguas turbias, Río de la Plata, SABIA-Mar, Ocean and Land Color Instrument.