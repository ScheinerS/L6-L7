%\begin{center}
%\large \bf \runtitle
%\end{center}
%\vspace{1cm}
\chapter*{\runtitle}

\noindent 
\textbf{Abstract}

Remote sensing in the optical region of the electromagnetic spectrum, or \textit{ocean color}, has demonstrated its ability to provide synoptic information of the optical and biogeochemical properties of the oceans. This is based on the determination of the spectral radiance that comes from the surface of the water, which is obtained from the signal that reaches the top of the atmosphere (TOA). The amplitude and spectral shape of this primary geophysical product (usually given as \textit {reflectance}) are interpreted in terms of derived products such as concentrations of optically active substances or IOPs (\textit{Inherent Optical Properties}) that can then be used in environmental applications and biogeochemical models at regional and global levels. The precision in estimating these parameters depends, however, on the ability to obtain the reflectance measured just above the surface of the water, $\rho_{w}$, from the total radiance measured by the sensor, $L_{TOA}$. The processing steps that must be applied to the signal include, among others, the elimination of the contribution of the atmosphere, a process called \textit{atmospheric correction} (AC). In this context, this thesis is aimed at exploring new alternatives of AC algorithms on the Río de la Plata Estuary (RdP), whose extremely turbid waters (mainly in the region of the turbidity front, at Barra del Indio) are a challenge both for traditional AC schemes, and for alternatives developed for turbid waters.

Chapter \ref{blr} describes the \textit{BLR-AC} algorithm (AC based on Baseline Residuals), which was calibrated, tested and validated on OLCI images (\textit{Ocean and Land Color Instrument}) although it is potentially applicable to other sensors with triples of spectrally close bands in the Red-NIR-SWIR range (NIR: \textit{Near Infrared}. SWIR: \textit{Short Wave Infrared}). The results show plausible performances compared to other schemes (such as BAC/BPAC v2.23, C2RCC, C2RCCnewNN, SeaDAS). Comparatively lower spatial correlations are observed between water and aerosol reflectances; better match-ups, and very good correspondences between the water reflectances on the Red/NIR estimated with BLR-AC and with a non-operational simple scheme - designed \textit{ad-hoc} for comparison - based on the extrapolation of the aerosol signal in fixed clear water windows to the entire region of interest. Although the results obtained with the BLR-AC scheme are plausible, it is still necessary to extend the correction to visible bands with an extrapolation method and possible additional boundary conditions in the visible region of the spectrum.

Chapter \ref{pca} describes the \textit{SWIR-PCA} (Principal Component Analysis of the aerosol signal using the SWIR bands) algorithm developed for optically complex waters of the Río de la Plata for sensors such as SABIA-Mar (Argentine-Brazilian Satellite for Sea Information) and MODIS (MODerate Resolution Imaging Spectroradiometer) that have bands in the spectral region of the far SWIR, that is, where the \textit{black pixel assumption} (Gordon 1978, \cite{gordon1978}) holds for any sediment concentration. The scheme is based on the decomposition into principal components of the atmospheric signal in the NIR/SWIR and the use of the bands in the SWIR for the determination of the signal in the NIR. The scheme was theoretically tested by means of simulations of radiative transfer of a water-atmosphere coupled system, showing plausible results in the determination of water reflectances in the NIR bands at the considered sensors (MODIS, SABIA-Mar, among others). In turn, a geostatistical approach applied to a set of Aqua/MODIS images was performed to test the impact of sensor noise on the performance of the algorithm, showing a low sensitivity of the scheme to sensor noise. Although the results of this scheme are plausible, it is still essential to test its performance on images of currently operative sensor, such as Aqua/MODIS and again extend the correction to visible bands.

Chapter \ref{ppe} describes a procedure for detecting and removing Prompt Particle Events (PPEs) that hit on sensors at LEOs (\textit{Low Earth Orbits}) - e.g. OLCI. Such events are extremely frequent in the South Atlantic Magnetic Anomaly region (SAA) - in particular, affecting the images of the Río de la Plata. The performance of this algorithm was evaluated both visually and by comparison with pre-existing results - evaluated on the OLCI sensor prior to launch. The results indicate that the images on the region of the SAA contain 27.8 times more pixels contaminated by PPEs than those corresponding to regions outside the SAA and that the most affected bands are those of 400 and 1016 nm, where the fraction of pixels where PPEs were detected over the SAA reached 0.14 \% and 0.26 \%, respectively. This correction is part of the preprocessing steps performed prior to the atmospheric correction described in chapter \ref{blr} (BLR-AC).

Chapter \ref{cam} describes an application developed for ocean color images of the Río de la Plata: the FAIT algorithm (\textit{Floating Algal Index for Turbid waters}) designed for floating vegetation (FV) detection in turbid waters. To perform this algorithm, the unusual invasion of water hyacinth (\textit{Eichhornia crassipes}) that occurred over RdP in January-April 2016 was considered from a multi-mission perspective. The impact of different spatial resolutions on the detection of FV was discussed, and finally its spatial extent, temporal variability and its relationship with the river's flow rate were analyzed and quantified.

Finally, in chapter \ref{oil} another specific application to these images is described: a possible algorithm for the detection of oil spills in turbid waters, together with various indices that might allow to determine the thickness of the surface layer of the spill. For this purpose, a sample of water from RdP was assembled on a laboratory scale to which different volumes of hydrocarbons were poured, and having exposed this sample to natural illumination conditions, water reflectances were measured at different spill thicknesses. The results show that there are several spectral indices that might be used as indicators of the presence of hydrocarbons in turbid waters, such as the ratio of red and infrared reflectances, luminosity, turbidity or the violet/red ratio. Although the differences found between oil-polluted and oil-free reflectances from RdP are well established, it is essential to expand this study to consider the impact of other relevant variables, such as the concentration of suspended particulate material, different types of hydrocarbons and different emulsification states.

\bigskip

\noindent\textbf{Keywords:} Ocean Colour, atmospheric correction, turbid waters, Río de la Plata, SABIA-Mar, Ocean and Land Colour Instrument.