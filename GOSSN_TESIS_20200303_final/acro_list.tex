\chapter{Listado de Acrónimos y Abreviaturas}
\label{acr}

\begin{acronym} % Give the longest label here so that the list is nicely aligned
\acro{ACOLITE}{Atmospheric Correction for OLI - lite (CA)}
\acro{AEB}{Agencia Espacial Brasileña}
\acro{AERONET}{AErosol RObotic NETwork (Red internacional de monitoreo de aerosoles)}
\acro{AOP}{Propiedad Óptica Aparente}
\acro{ASD}{Analytic Spectral Devices (ASD FieldSpec FR) (Radiómetro de campo)}
\acro{AUS}{Región noroeste del Mar de Australia (\S \ref{ppe})}
\acro{AVIRIS}{Airborne Visible/Infrared Imaging Spectrometer}
\acro{BAC/BPAC}{Baseline Atmospheric Correction/Bright Pixel Atmospheric Correction (CA)}
\acro{BBl}{Bahía Blanca (Buenos Aires, Argentina)}
\acro{BE}{Costa Belga}
\acro{BL}{Línea de Base}
\acro{BLR}{Resíduo de la Línea de Base}
\acro{BLR-AC}{Corrección atmosférica por BLRs}
\acro{BPC}{Bright Pixel Correction}
\acro{BRDF}{Función de Distribución Bidireccional de la Reflectancia}
\acro{C2RCC}{Case 2 Regional Processor (CA)}
\acro{C2RCCnewNN}{íd. C2RCC, con nueva red neuronal}
\acro{CA}{Corrección Atmosférica}
\acro{CABA}{Ciudad Autónoma de Buenos Aires}
\acro{CCD}{Dispositivo de Carga Acoplada}
\acro{CDOM}{Material Orgánico Disuelto Coloreado}
\acro{CEILAP-BA}{Centro de Investigaciones en Láseres y Aplicaciones - Buenos Aires (Estación de AERONET)}
\acro{CENPAT}{CEntro Nacional PATagónico}
\acro{CHL}{Clorofila-a}
\acro{CHRIS/PROBA}{Compact High Resolution Imaging Spectrometer/}
\acro{CIE}{Commission Internationale de l'Eclairage (cf. espacio CIE-La*b*, \S \ref{cam})}
\acro{CNES-SOS}{Centre National d'Études Spatiales-Successive Orders of Scattering (código de TR)}
\acro{CODA}{Copernicus Online Data Access}
\acro{CONAE}{Comisión Nacional de Actividades Espaciales}
\acro{CONICET}{Consejo Nacional de Investigaciones Científicas y Técnicas}
\acro{CR-800}{\textit{Datalogger} de Campbell Scientific (utilizado para el OBS501)}
\acro{CRBasic}{Lenguaje de comandos a instrumentos de Campbell Scientific (por ej., OBS501)}
\acro{CTC}{Contenido Total de Columna de Aire}
\acro{CV}{Coeficiente de Variación}
\acro{CZCS}{Coastal Zone Color Scanner}
\acro{DQL}{Data Query Language (Lenguaje de Consulta de Datos)}
\acro{DRG}{Región de dragado (\S \ref{cam})}
\acro{DU}{Unidades Dobson}
\acro{ECMWF}{European Centre for Medium-Range Weather Forecasts}
\acro{EETR}{Ecuación Escalar de Transferencia Radiativa}
\acro{EPEA}{Estación Permanente de Estudios Ambientales}
\acro{EPV}{Evento de Partícula Veloz}
\acro{ESA}{Agencia Espacial Europea}
\acro{ETR}{Ecuación de Transferencia Radiativa}
\acro{EVTR}{Ecuación Vectorial de Transferencia Radiativa}
\acro{FAI}{Índice de Algas Flotantes}
\acro{FAIT}{Índice de Algas Flotantes para aguas Turbias}
\acro{FBU}{Formazin Backscattering Units}
\acro{FLH}{Altura de la Línea de Fluorescencia}
\acro{FNU}{Formazin Nephelometric Units}
\acro{FOV}{Field-Of-View}
\acro{GCOS}{Global Climate Observing System}
\acro{GF/F}{Filtros de Fibra de Vidrio}
\acro{GOCI}{Geostationary Ocean Color Imager}
\acro{HACH}{Turbidímetro portátil de HACH, \S \ref{dat}}
\acro{HICO}{Hyperspectral Imager for the Coastal Ocean}
\acro{HyspIRI}{Hyperspectral Infrared Imager}
\acro{IAFE}{Instituto de Astronomía y Física del Espacio}
\acro{INPE}{Instituto Nacional de Pesquisas Espaciais}
\acro{IOP}{Propiedades Ópticas Inherentes}
\acro{IPR}{Rango interpercentil}
\acro{IQR}{Rango intercuartil}
\acro{IR}{Infrarrojo}
\acro{ISO}{Organización Internacional de Estándares}
\acro{ITR}{Rango intertercil}
\acro{LandSat}{Land Satellite (serie de misiones de NASA y USGS)}
\acro{L8}{LandSat-8/OLI}
\acro{LED}{Diodo Emisor de Láser}
\acro{LEO}{Low Earth Orbit}
\acro{LUT}{Look-Up-Table}
\acro{MA}{Aqua/MODIS}
\acro{MAD}{Desviación Absoluta Mediana}
\acro{MAE}{Error Absoluto Medio}
\acro{MCI}{Índice de Clorofila Máxima}
\acro{MERIS}{MEdium Resolution Imaging Spectrometer}
\acro{MODIS}{MODerate-resolution Imaging Spectroadiometer (a bordo de Aqua y Terra)}
\acro{MSDA XE}{Software de procesamiento del TriOS}
\acro{MSI}{Multispectral Instrument (a bordo de Sentinel-2)}
\acro{MT}{Terra/MODIS}
\acro{NASA}{National Aeronautics and Space Administration}
\acro{NDVI}{Índice de Vegetación por Diferencia Normalizada}
\acro{NIR}{Infrarrojo cercano}
\acro{NaN}{Not-A-Number}
\acro{OAA}{Ángulo de Observación Acimutal}
\acro{OBPG}{Ocean Biology Processing Group}
\acro{OBS}{íd. OBS501}
\acro{OBS501}{Medidor de turbidez de Campbell Scientific, \S \ref{dat}}
\acro{OBS-BS}{Retrodispersómetro del OBS501}
\acro{OBS-SS}{Nefelómetro del OBS501}
\acro{OCM}{Indian Ocean Colour Monitor}
\acro{OLCI}{Ocean and Land Colour Instrument, a bordo de Sentinel-3}
\acro{OLI}{Ocean and Land Instrument, a bordo de LandSat-8}
\acro{OOB}{Out-Of-Band}
\acro{OZA}{Ángulo cenital de Observación}
\acro{PACE}{Plankton, Aerosol, Cloud, ocean Ecosystem (misión satelital de la NASA)}
\acro{PC}{Ficocianina}
\acro{PCA}{Análisis por Componentes Principales}
\acro{POLYMER}{Esquema de CA de espectro completo}
\acro{PSD}{Distribución de Tamaños de Partículas}
\acro{qSSA}{Aproximación de Dispersión cuasi-Simple}
\acro{RAA}{Ángulo Acimutal Relativo}
\acro{RC}{Corregido por dispersión Rayleigh}
\acro{REM}{Radiación Electromagnética}
\acro{RGB}{Rojo-Verde-Azul}
\acro{RMSD}{Raíz de la Diferencia Cuadrática Media}
\acro{RNS}{Rojo/NIR/SWIR}
\acro{ROI}{Región de Interés}
\acro{RdP}{Río de la Plata}
\acro{Sentinel}{Serie de misiones satelitales de la ESA}
\acro{S2}{Sentinel-2/MSI}
\acro{S3}{Sentinel-3/OLCI}
\acro{SAA}{Anomalía del Atlántico Sur (\S \ref{ppe})}
\acro{SABIA-Mar}{Satélite Argentino Brasileño para Información del Mar}
\acro{SAOCOM}{Satélite Argentino de Observación Con Microondas}
\acro{SAR}{Radar de Apertura Sintética}
\acro{SCI}{Índice Sintético de Clorofila}
\acro{SMAR}{íd. SABIA-Mar}
\acro{SNR}{Relación Señal-Ruido}
\acro{SOLSPEC}{Solar Spectrometer, a bordo de SOLAR}
\acro{SOM}{Concentración de Material Orgánico Disuelto}
\acro{SOS}{íd. CNES-SOS}
\acro{SPM}{Concentración de Material Particulado en Suspensión}
\acro{SQL}{Structured Query Language (lenguaje de programación tipo DQL)}
\acro{SRF}{Función de Respuesta Espectral}
\acro{SVC}{Coeficientes de Calibración Vicaria}
\acro{SWIR}{Infrarrojo de Onda Corta}
\acro{SZA}{Ángulo Cenital Solar}
\acro{SeaDAS}{SeaWIFS Data Analysis System}
\acro{SeaWIFS}{Sea-Viewing Wide Field-of-View Sensor}
\acro{TOA}{Top-Of-Atmosphere (Tope de la Atmósfera}
\acro{TOMS/EPTOMS}{Total Ozone Mapping Spectrometer onboard the Earth Probe spacecraft}
\acro{TR}{Transferencia Radiativa}
\acro{TTR}{Teoría de Transferencia Radiativa}
\acro{TW}{Región de agua turbia (\S \ref{cam})}
\acro{TriOS}{Radiómetro de campo (\S \ref{dat})}
\acro{UA}{Unidad Astronómica}
\acro{USGS}{United States Geological Survey}
\acro{UTC}{Tiempo Universal Coordinado}
\acro{UV}{Ultra Violeta}
\acro{VCCS}{Voltage-Controlled Current Source}
\acro{VF}{Vegetación Flotante}
\acro{VIIRS}{Visible Infrared Imaging Radiometer Suite, a bordo de Suomi-NPP}
\acro{VIS}{Visible}
\acro{VSF}{Función de Dispersión en Volumen}
\acro{WC}{Whitecaps (Espuma)}
\acro{WMO}{Organizacióm Meteorológica Mundial}
\acro{XTW}{Región de agua extremadamente turbia (\S \ref{cam})}
\end{acronym}

